%************************************************
\chapter{Living Wage Gap}\label{ch:living_wage_gap}
%************************************************

This section will start to look at the living wage gap. First, we will come up with two definitions for the living wage gap, one based on the median wage, and one defined by the minimum wage. After that, we will look at distributions of both definitions. Finally, we will look at the distribution of households who earn the living wage or below.

\section{Distribution of The Median-Gap}

Figure \ref{f:ch5_median_gap_years} shows the living wage gap distribution across counties when the gap is between the living wage the county's median wage, as defined by the Small Area Income and Poverty Estimate data set described earlier. This gap has maintained stability over time, fluctuating around \$30,000. This means, for those making the median wage in their county, on average, they have \$30,000/year left over above and beyond their living costs.

\begin{figure}[H]
    \centering
        \includegraphics[width=\textwidth]{median_gap_years.png}
        \caption{Weighted Distribution of the Gap Between County Median Wage and Living Wage (2004-2014)}
    \label{f:ch5_median_gap_years}
\end{figure}



\section{Distribution of The Minimum-Wage-Gap}

Figure \ref{f:ch5_minimum_gap_years} shows a similar figure to Figure \ref{f:ch5_median_gap_years}, but for the living wage gap for the minimum wage. Using the minimum wage data discussed in the previous chapter, we find the difference between the county's applicable minimum wage and its minimum wage. A line showing the break even point is shown, with the green plots showing what the living wage gap would look like with a \$15.00/hour minimum wage for that year (inflation causes that \$15 to be worth less over time, which explains their trend towards the left over time). The figure shows that this gap seems to have peaked in 2007, but the gains made from 2007 - 2010 are being eaten away again. The distribution in 2014 is very similar to 2008, and if the trend continues, will match the peak in 2007. Future extensions to this project should look into the most recent data available to see how the trend continues.

\begin{figure}[H]
    \centering
        \includegraphics[width=\textwidth]{minimum_gap_years.png}
        \caption{Distribution of the Gap Between County Minimum Wage and Living Wage (2004-2014)}
    \label{f:ch5_minimum_gap_years}
\end{figure}


\section{Percentage of Single Households At or Below the Living Wage}

Using the wage distribution data from the Census, we can see what percentage of people in a county are making the living wage or less. Figure \ref{f:ch5_perc_below_living} shows this distribution as a histogram and a violin plot. A note on methodology: this uses the non-family household column in the data to match our model definition, since this model is for single adults with no children. Also the data from the census only breaks down the wage distribution based on their own bucketing. This percentage therefore is an approximation that can err on the higher or lower side. In the figure, the green distribution is the under-estimate, while the blue distribution is the over estimate, with the real distribution being somewhere in the middle. Neither prints a pretty picture, as there seems to be significant portions of the population that are making the living wage or below. To see details of how this was generated, please consult the associated IPython notebook.

Figure \ref{f:ch5_regional_below_living} shows how this breaks down over region, by showing the percentage of counties in a region that have more than 50\% of their constituents making the living wage or less. This was done using the over-estimated values; using the under-estimated values and changing the threshold to the living wage at the peak (approx \$30,000) creates a similar pie chart.

The last figure shows the percentage of people in a county making the living wage or less, distributed across the map of the united states. This is an interesting plot, as it shows that areas in the deep south have a very high proportion of people in a precarious situation. In contrast to Figure \ref{f:ch4_county_map_2014}, these areas don't have very high living wages, which shows that these areas might be plagued with such low wages that many people cannot make enough to get by.

\begin{figure}[H]
    \centering
        \includegraphics[width=\textwidth]{perc_below_living.png}
        \caption{Estimated Distribution of Percentage of Single Households Earning the Living Wage or Less Across Counties}
    \label{f:ch5_perc_below_living}
\end{figure}

\begin{figure}[H]
    \centering
        \includegraphics[width=0.80\textwidth,height=0.40\textheight]{regional_below_living.png}
        \caption{Regional Breakdown of Counties that have 50\% or Greater Single Households that make the Living Wage or Less}
    \label{f:ch5_regional_below_living}
\end{figure}

\begin{figure}[H]
\includegraphics[width=\textwidth,height=0.40\textheight]{map_perc_below.png}
    \centering
        \caption{Map of Counties Showing Percentage of Single Households Earning The Living Wage or Less}
    \label{f:ch5_map_perc_below}
\end{figure}