%************************************************
\chapter{Results and Future Work}\label{ch:results}
%************************************************

\section{Overall Results}

Here is a summary of points visited in the previous sections:

\begin{itemize}
\item The living wage seems to have levelled off in last few years. Not sure why, but future analysis would be interesting

\item Gap between minimum wage and the living wage reached a peak, and then some gains were made due to increases in the minimum wage. However, due to rent and inflation, current levels are close to the peak again.

\item The top 150 most populous counties have a much higher living wage than the rest of the country. This is mostly due to rent being higher in densely populated areas, as rent is the model variables with the highest impact.

\item The top 150 most populous counties can be split into two groups (bi-modal). This shows that even in the top 150 counties by population, there are a handful of cities with a very high living wage. 

\item White and Blacks seem to have similar living wage distributions, as they are relatively well mixed across the county. Other races, especially Asians, Hispanics and Pacific Islanders, are concentrated into areas with high living wages. Population dynamics plays a huge role in determining the living wage.

\item Races seem to experience the same increases over time, with their distributions being controlled by population dynamics.

\item The gap between a county's median wage and it's living wage seems stable over time. This might indicate that those making a median wage for their area are not so concerned about the living wage. Analysis of the gap between the living wage and the minimum wage show that the minimum wage in all areas is not enough to support a living wage for a single adult household.

\item When looking at \textbf{only} single households, we see that on average, 30\% - 50\% of them are making the living wage or less. This range is due to inaccuracies in determining the wage distribution per county, but still shows that many areas in the country have significant portions of the population 'living on the edge'.

\item The region that has the most counties that have 50\% or more of their single households making he living wage or less, is the South.

\end{itemize}



\section{Future Work}

\begin{itemize}
\item If we could get wages broken down by race \textbf{and} by county, this would allow us to see how the living wage gap have evolved over time between races. 

\item Data that supports the model variables but at a more granular level would help increase the model's accuracy overall.

\item Better data regarding wage distributions per county would help tighten up the analysis of those making the living wage or less.

\end{itemize}