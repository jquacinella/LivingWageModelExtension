%************************************************
\chapter{Results and Future Work}\label{ch:results}
%************************************************

\section{Overall Results}

Will expand on each point:

\begin{itemize}
\item The living wage seems to have levelled off in last few years. Not sure why, but future analysis would be interesting
\item Gap between minimum wage and the living wage reached a peak, and then some gains were made due to increases in the minimum wage. However, due to rent and inflation, current levels are close to the peak again.
\item The top 150 most populous counties have a much higher living wage than the rest of the country. This is mostly due to rent
\item The top 150 most populous counties can be split into two groups. 
\item White and blacks seem to have similar living wage distributions, as they are relatively well mixed across the county. Other races, especially asians and PI, are concentrated into areas with high living wages.
\item Races seem to experience the same increases over time, with their distributions being controlled by population dyamics (since the living wage changes most with location)
\item When looking at **only** single households, we see that on average, 50% of them are making the living wage or less.

\item Median wage?
\end{itemize}



\section{Future Work}

\begin{itemize}
\item If we could get wages broken down by race **and** by county, this would allow us to see how the living wage gap have evolved over time between races. 
\end{itemize}