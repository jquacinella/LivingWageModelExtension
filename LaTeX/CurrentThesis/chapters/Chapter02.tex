%************************************************
\chapter{Data Sources and Collection}\label{ch:data_collection}
%************************************************

This section will outline how data was gathered for the various model parameters, as well as other data we need to calculate their values. The original model was made for 2014 data and extending this data to the past means we need to be careful that any changes in the underlying data methodology of these parameters needs to be noted. All data files mentioned here are available in the github repository, under the data/ directory. Each data source is typically loaded into a Pandas DataFrame, which can be seen in the code sections linked to in the Appendix, or via the associated IPython notebook. 

\section{Consumer Expenditure Report}

The Consumer Expenditure Report\cite{cex_survey} is used by the living wage model to determine 3 variables in the model, $transportation\_cost$, $health\_care\_costs$, and $other\_necessities\_cost$. From their website:

\begin{quote}
The Consumer Expenditure Survey (CE) program consists of two surveys, the Quarterly Interview Survey and the Diary Survey, that provide information on the buying habits of American consumers, including data on their expenditures, income, and consumer unit (families and single consumers) characteristics ... The CE is important because it is the only Federal survey to provide information on the complete range of consumers' expenditures and incomes, as well as the characteristics of those consumers.
\end{quote}

After downloading, the specific data needed for the model variables were extracted by hand. All data files are stored under the cex\_survey subdirectory. 

\section{USDA Food Plans}

The Cost of Food project from the USDA\cite{usda} produces different food plans (The Thrifty, Low-Cost, Moderate-Cost, and Liberal Food Plans), which represent a nutritious diet at different costs. This dataset determines one variable in the model, $food\_cost$. Also, the original model uses regional weighting factors to better model varying food prices across the county. \cite{usda_regional} After downloading, the specific data needed for the model variables were extracted by hand. All data files are stored under the food subdirectory. 

\section{Free Market Rent Data From HUD}

The U.S. Department of Housing and Urban Development produces the Fair Market Rent dataset, which the model uses as 'gross rent estimates' for the $housing\_cost$ variable. \cite{fmr} After downloading, the specific data needed for the model variable were extracted by hand, using the FMR1 column as the best estimate for the housing costs associated with a 1-bedroom apartment. All data files are stored under the housing\_cost subdirectory. 

\section{Most Populated Counties}

An article from Business Insider lists the top 150 counties by population. \cite{populated_counties} This project uses this to determine if there are systemic differences between the living wage with respect to county population.

\section{Medical Expenditure Panel Survey from the AHRQ}

The original model uses data from the Medical Expenditure Panel Survey (MEPS), which is done by the The Agency for Healthcare Research and Quality. From their website \cite{meps}:

\begin{quote}
The Medical Expenditure Panel Survey, which began in 1996, is a set of large-scale surveys of families and individuals, their medical providers (doctors, hospitals, pharmacies, etc.), and employers across the United States. 
\end{quote}

This data is used for $insurance\_premiums$ model variable. All data files are stored under the insurance subdirectory. 

\section{Tax Data}

The following data sources are used in calculating the $tax\_rate$ model variable.

\subsection{Payroll Taxes}

Payroll tax data was manually downloaded from the Social Security and Medicare Tax Rates web page from the Social Security Administration website. \cite{ssa} 

\subsection{State Tax Data}

The Tax Foundation produces a spreadsheet of official State income tax rates. This spreadsheet is not in a useful format for analysis, so data was manually copied to formatted\_state\_taxes.csv file under the taxes subdirectory. \cite{state_tax}

\subsection{Federal Tax Data}

The Tax Policy Center produces a dataset called the "Historical Federal Income Tax Rates for a Family of Four". While this dataset is not quite what this model needs, since the model developed for this project only modeled single adult households, due to a lack of data for single households (which only goes back to 2011), this dataset is used. \cite{federal_tax} Since all counties experience the same federal tax rate, inaccuracies here would not affect overall trends, but produce a worse approximation to the living wage consistently across counties.

\section{Race Data from 2010 Census}

Using the Census' "American Fact Finder" web page, data on racial breakdowns per county were downloaded from the "DP-1: Profile of General Population and Housing Characteristics: 2010" dataset. \cite{race_data} Code for loading this data can be found in the IPython notebook. \cite{code_data_race}


\section{Median Wage Data}

For the median wage, there are two data sets from the Census.gov that are applicable. The American Community Survey provides this data, however this project will use the SAIPE data for estimates of county median income:

\begin{quote}
Small Area Income and Poverty Estimates: The SAIPE program produces single-year estimates of median household income and poverty for states and all counties, as well as population and poverty estimates for school districts. Since SAIPE estimates combine ACS data with administrative and other data, SAIPE estimates generally have lower variance than ACS estimates but are released later because they incorporate ACS data in the models. For counties and school districts, particularly those with populations below 65,000, the SAIPE program provides the most accurate sub-national estimates of poverty. For counties, SAIPE generally provides the best single year estimates of median household income.
\end{quote}


\section{Minimum Wage}

For the minimum wage, The United States Department of Labor keeps track of state level minimum wages. \cite{minimum_wage_data} The data is loaded via the IPython notebook. \cite{minimum_wage_load} This data was created by hand, and is located under the census/MinimumWage subdirectory in the github repository. 

\section{Wage Distribution}

It will also be instructive to see any data regarding the wage distribution per county, to get an estimate of how many people are earning a wage at or below the living wage. Data was loaded from American Fact Finder on the Census website \cite{wage_data} and is stored in the wage\_distribution subdirectory on github.